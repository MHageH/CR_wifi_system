\documentclass[11pt]{article}
\renewcommand{\baselinestretch}{1.05}

\usepackage{amsmath,amsthm,verbatim,amssymb,amsfonts,amscd, graphicx}
\usepackage{graphics}

\usepackage{xcolor}

\usepackage[hidelinks]{hyperref}
\usepackage{parskip}

\usepackage{subcaption}
\usepackage{wrapfig}
\usepackage{multirow}

\renewcommand{\contentsname}{Table des mati\`eres}
\renewcommand\refname{R\'ef\'erences}

\topmargin0.0cm
\headheight0.0cm
\headsep0.0cm
\oddsidemargin0.0cm
\textheight23.0cm
\textwidth16.5cm
\footskip1.0cm

\begin{document}

\title{\textbf{Projet de Conception d'un syst\`eme\\ de communication WiFi} \\ \'Etude de la partie syst\`eme}
\author{Cl\'ement Cheung \\ Mohamed Hage Hassan \medskip\\\medskip \textbf{Encadrant} \medskip \\ Yannis Le Guennec}
\date{21 D\'ecembre 2017}
\maketitle
\thispagestyle{empty}

\renewcommand{\abstractname}{Pr\'eambule}

\begin{abstract}

\end{abstract}

\vskip 5cm

\begin{figure}[!htb]
\begin{center}
  \includegraphics[scale=1.2]{phelma-logo.jpg}
\end{center}
\end{figure}

\begin{center} \textbf{Institut Polytechnique de Grenoble} \end{center}

\clearpage

\tableofcontents
\clearpage

\iffalse

\begin{figure}[!htb]
\begin{center}
  \includegraphics[scale=0.47]{Echantillonneur-bloqueur.png}
  \caption{Sch\'ema d'un \'echantilloneur-bloqueur \`a capacit\'e commut\'ee}
\end{center}
\end{figure}

\begin{figure}[!htb]
  \begin{subfigure}[t]{.5\linewidth}
      \centering
      \includegraphics[width=1.1\linewidth]{circuit-RC.png}
      \label{fig:rccircuit}
  \end{subfigure}%
  \begin{subfigure}[t]{.5\linewidth}
    \centering
    \includegraphics[width=1.1\linewidth]{sim-inital.png}
    \label{fig:rccircuit-sim}
  \end{subfigure}%
  \caption{Sch\'ema et Simulation du circuit}
  \label{fig:RC-sim}
\end{figure}

\fi


\section{Introduction}
\section{Objectifs}

\section{Analyse du signal Wifi}

\section{Transmetteur Wifi}
\subsection{M\'elangeur RF}
\subsection{Filtre RF}
\subsection{\'Etude de la non-lin\'earit\'e}

\section{R\'ecepteur Wifi}
\subsection{Blocs de base}
\subsection{Sensibilit\'e du r\'ecepteur}
\subsection{Point d'intermodulation d'ordre 3 du r\'ecepteur IIP3}
\subsection{SFDR}
\subsection{Dynamique du r\'ecepteur sans bloqueur}
\subsection{Dynamique avec un bloqueur}

\section{Cannal de propagation Radio}
\subsection{Consid\'eration de la propagation en espace libre}
\subsection{\'Etude de la propagation pour un cannal Wifi}

\section{Conclusion}

\clearpage
\section{Annexes}


\clearpage

\addcontentsline{toc}{section}{R\'ef\'erences}

\begin{thebibliography}{9}
\bibitem{Projets-Analog}
\textit{Projets de conception en Micro\'electronique analogique}\\
\texttt{Laurant Aubard, Fatah Rarbi, Daniel Zahini, Florent Cilici}\\
\texttt{Institut Polytechnique de Grenoble - Phelma}

\end{thebibliography}


\end{document}
